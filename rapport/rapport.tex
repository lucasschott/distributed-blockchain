\documentclass[a4paper,11pt,DIV=12]{scrreprt}
\usepackage[utf8]{inputenc}
\usepackage[T1]{fontenc}
\usepackage[francais]{babel}
\usepackage{graphicx}
\usepackage[export]{adjustbox}
\usepackage{setspace}
\usepackage{dirtytalk}
\usepackage{amsmath}
\usepackage[lined,boxed,commentsnumbered, ruled,vlined,linesnumbered, french]{algorithm2e}
\usepackage{algorithmic}
\usepackage{bytefield}
\usepackage{xcolor}
\usepackage{listings}
\definecolor{codegray}{rgb}{0.92,0.92,0.92}

\lstset{basicstyle=\ttfamily\small,
backgroundcolor=\color{codegray},   
showstringspaces=false,
commentstyle=\color{red},
keywordstyle=\color{blue}
}

\renewcommand{\algorithmicrequire}{\textbf{Require:}}
\renewcommand{\algorithmicensure}{\textbf{Ensure:}}
\renewcommand{\algorithmiccomment}[1]{\{#1\}}
\renewcommand{\algorithmicend}{\textbf{fin}}
\renewcommand{\algorithmicif}{\textbf{si}}
\renewcommand{\algorithmicthen}{\textbf{alors}}
\renewcommand{\algorithmicelse}{\textbf{sinon}}
\renewcommand{\algorithmicelsif}{\algorithmicelse\ \algorithmicif}
\renewcommand{\algorithmicendif}{\algorithmicend\ \algorithmicif}
\renewcommand{\algorithmicfor}{\textbf{pour}}
\renewcommand{\algorithmicforall}{\textbf{pour tout}}
\renewcommand{\algorithmicdo}{\textbf{faire}}
\renewcommand{\algorithmicendfor}{\algorithmicend\ \algorithmicfor}
\renewcommand{\algorithmicwhile}{\textbf{tant que}}

\title{Rapport de projet: Blockchain distribuée}
\author{Lucas SCHOTT}


\begin{document}

%\begin{onehalfspace}

    \maketitle

    \chapter{Introduction}

    \section{Présentation}

    Le but de ce projet est de réaliser une blockchain distibuée et de pouvoir
    réaliser une simulation d'utilisation de la blockchain pour pouvoir observer
    son évolution dans le temps.

    Dans cette blockchain distribuée, les noeuds bloc stockent la blockchain,
    receptionnent des demandes de transactions et mine des block pour valider
    ces transactions. Les noeuds bloc communiquent entre eux pour s'échanger les
    nouveaux block, ou envoyer une blockchain entière.
    Les noeuds participants font des demandes de transactions, et peuvent
    demander aux noeuds bloc la quantité de points blockchain dont ils disposent.

    Ce projet à été réalisé en C++ en utilisant les sockets et la bibliothèque
    de programamtion distribuée RCF. Il est uniquement utilisable avec des adresse IPv6.

    \section{fonctionnement}

    Un participant peut contacter un noeud bloc et lui envoyer une demande de
    transacation d'un compte a un autre, le noeud bloc met alors la transaction
    en attente pour ensuite la miner dans un block. Le participant peut demander a un noeud bloc quelle est la
    quantité de points blockchain dont dispose un compte, c'est ce qu'on
    appelle la "balance", elle peut etre négative si un participant effectue une
    transaction supérieur à ce dont il dispose.

    Les noeud blocs attendent d'avoir suffisement de transactions en attente
    pour ensuite les miner dans un block et valider les transactions. Il lui faut 5
    transactions en attente pour commencer le minage. Une fois le block
    miné, le noeud bloc va l'envoyer à tous les autres noeuds bloc auquel il
    est connecté. Un noeud bloc ne peux miner qu'un block a la fois, donc toutes
    les transactions recu pendant un minage sont misent en attente jusqu'au
    minage du prochain bloc.
    \\

    Plusieurs noeuds bloc peuvent se connecter entre eux, au moment de la
    connection de deux blocs ceux-ci s'échangent les blockchain qu'ils
    possèdent. Le noeud bloc garde la blockchain valide la plus longue.
    Ensuite les noeuds bloc s'envoient mutuellement les listes des
    autres noeuds bloc qu'il connaissent, et les noeuds blocs vont ainsi démarrer de nouvelles
    procédure de connexion avec les autres noeuds bloc. Tous les noeaud blocs sont
    connectés entre eux avec une topologie full-mesh. Ainsi quand un noeud bloc
    mine un nouveau block, il va le transmettre à tous les autres noeuds bloc. 
    Si un noeud bloc recoit 3 block invalides de suite il va contacter un autre
    noeud bloc pour lui demander sa blockchain, comme ca il pourra avoir à
    nouveau la bonne blockchain à condition que celle-ci soit plus longue que
    la sienne.

    Les noeuds bloc connectés s'envoient des messages de manière périodique pour
    maintenir leur connection, si un noeud bloc ne reçoit plus de messages d'un
    autre noeud bloc il va le conséderer comme mort et les deux noeuds bloc se
    déconnecteront. Un noeud bloc est considéré comme mort au bout de 2 secondes
    sans réponse. Le réseau de noeud bloc est résistants aux pannes, aux
    déconnexions et dysfonctionnements de certains noeuds bloc. 

    \chapter{Utilisation}

    \ \\
    Les détails concernant l'intallation du programme sont disponible dans le fichier README.
    \\


    \begin{lstlisting}
    $ ./bloc <compte> <adresse_IPv6|nom_de_domaine> <numero_de_port> [-v]
    \end{lstlisting}

    Sert à creer un noeud bloc unique, avec un compte pour récupere les
    récompenses du à son travail de minage, en lui donnant aussi une adresse IPv6 à utiliser
    disponible sur la machine courante et un numéro de port à utiliser, 
    ce noeud bloc peut recevoir des demandes de transactions 
    \\ \\

    \begin{lstlisting}
    $ ./bloc <compte>  <adresse|nom_de_domaine> <numero_de_port>
    <adresse|nom_de_domaine> <numero_de_port> [-v]
    \end{lstlisting}

    Permet de creer un noeud bloc comme précédement et de le connecter à un autre
    noeud bloc déjà existant, en lui donnant en plus de sa propre adresse et
    numéro de son port, l'adresse (ou le nom de domaine) et
    le numéro de port du server auquel se connecter. Le nouveau noeud bloc
    communiquera avec celui déjà en place et intègrera le groupe de noeuds bloc
    dans lequel le noeud bloc contacté est déjà intégré.
    \\ \\

    \begin{lstlisting}
    $ ./participant <adresse_IPv6|nom_de_domaine> <port>
    <comtpe_source> <compte_destination> <montant>[-v]
    \end{lstlisting}

    Permet a un participant de contacter un noeud bloc en lui faisant une demande
    de transaction du compte source au compte destination, du montant donné.
    \\ \\

    \begin{lstlisting}
    $ ./participant  <adresse_IPv6|nom_de_domaine> <port>
    balance <compte> [-v]
    \end{lstlisting}

    Permet au noeud participant de demander au noeud bloc la balance du compte spécifié(nombre de points blockchain).
    \\ \\

    Les deux programmes, \emph{participant} comme \emph{bloc} peuvent prendre
    l'option \emph{-v} (verbose) en argument. Avec cette option activée le
    programme affiche explicitement ses différentes actions, et plus
    particulièrement les détails de synchonisations entre server, et les
    émissions et réceptions de messages.

    Si le prgramma \emph{bloc} est utilisé avec l'option -d (pour display)
    celui ci se connectera au noeud bloc indiqué et récuperera la blockchain
    pour l'afficher sur la sortie standard, puis quittera.

    \chapter{Format de messages}


    Il y a quatre formats de messages envoyés via les sockets:
    \begin{itemize}
        \item Le format \emph{transaction\_t} est utilisé quand un participant ou un noeud bloc envoi un
            transaction à un autre noeud bloc.
            \\
            \\

            \begin{bytefield}{32}
                \bitheader[lsb=0]{0,7,8,15,16,23,24,31,32} \\
                \bitbox{16}{2 octets: en-tête} & \bitbox[lrt]{16}{} \\
                \wordbox[lr]{1}{50 octets: compte source} \\
                \skippedwords \\
                \wordbox[lrb]{1}{} \\
                \wordbox[lr]{1}{50 octets: compte destination} \\
                \skippedwords \\
                \wordbox[lrb]{1}{} \\
                \wordbox[lrb]{1}{4 octets: montant de la transaction} \\
            \end{bytefield}

        \item Le format \emph{ask\_balance\_t} est utilisé quand un participant
            demande sa balance à un noeud bloc.
            \\
            \\

            \begin{bytefield}{32}
                \bitheader[lsb=0]{0,7,8,15,16,23,24,31,32} \\
                \bitbox{16}{2 octets: en-tête} & \bitbox[lrt]{16}{} \\
                \wordbox[lr]{1}{50 octets: compte} \\
                \skippedwords \\
                \wordbox[lrb]{1}{} \\
            \end{bytefield}

        \item Le format \emph{send\_balance\_t} est utilisé quand un noeud
            bloc envoi le montant de la balance d'un participant à celui ci.
            \\
            \\

            \begin{bytefield}{32}
                \bitheader[lsb=0]{0,7,8,15,16,23,24,31,32} \\
                \bitbox{16}{2 octets: en-tête} & \bitbox[lrtb]{16}{} \\
                \bitbox[lrb]{16}{4 octets: montant } \\
            \end{bytefield}

        \item Le format \emph{syn\_t} est utilisé quand un noeud
            bloc envoi des messages de connexion ou de synchronisation avec un
            autre noeud bloc.
            \\
            \\

            \begin{bytefield}{32}
                \bitheader[lsb=0]{0,7,8,15,16,23,24,31,32} \\
                \bitbox{16}{2 octets: en-tête} & \bitbox[lrt]{16}{} \\
                \wordbox[lr]{1}{16 octets: adresse noeud bloc } \\
                \skippedwords \\
                \wordbox[lrb]{1}{} \\
                \bitbox[lrb]{32}{4 octets: port} \\
            \end{bytefield}

    \end{itemize}

     \\ \\
    Les autres informations et données échangés : les blockchain et les blocks,
    le sont via la bibliothèque RCF.


    \chapter{Implémentation}


    \section{Réseau et Distribution}

    \subsection{Transmission des transactions}

    Les transaction sont envoyés par un noeud participant à un noeud bloc via
    des sockets. À la réception de cette requete le noeud bloc, va mettre en
    attente la transaction avant minage du prochain block.

    La demande d'un client de la balance de son compte à un noeud bloc se fait
    via les sockets. À la réception d'une telle requete le noeud bloc, va calculer
    la balance du compte, et la retourner un noeud participant via des socket.


    \subsection{Échanges de block et de blockchain}

    Les noeuds blocs s'échangent des blocks via un appelle à une méthode d'une
    interface partagé par un autre noeud bloc en utilisant la bibliothèque de
    programmation distribué RCF.
    Les échange de blockchain se font de la même manière.


    \subsection{Connexion des noeuds bloc}

    Les noeud blocs se connectent entre eux via des sockets. Ils s'envoient
    périodiquement des messages keep alive pour maintenir leur connection. 

    \section{blobkchain}

    La Blockchain est un objet contenant un vecteur de Blocks, une récompense de minage,
    une difficulté de minage et des transaction en attentes qui ne sont pas
    encore dans la blockchain.
    Un Block est un objet contenant le hash du block précédent, une date, un
    nonce, un vecteur de Transactions, et un hash.
    Une Transaction est un objet contenant un compte source, un comptee destination et un montant.

    A partir de 5 transactions en attente le noeud bloc va miner un block pour
    l'ajouter dans la blockchain. Les transactions qu'il recevera en même temps
    seront mis en attente pour le prochain bloc. Si un noeud bloc recoit 3 block
    invalides de suite il va demander une nouvelle blockchain à une autre noeud bloc.
    Le noeud bloc va comparer ces deux chaine jusqu'au moment ou elle divergent,
    a partir de la le noeud bloc va mettre toutes les transactions effectués dans
    les block suivant de la blockchain la plus courte dans les transactions en attente.
    Il va ensuite pouvoir les miner à la suite de la nouvelle blockchain mise à jour.



    \chapter{Conclusion}


    Le sujet du projet était très intéréssant, Il m'a  appris énormément sur le
    programmation distribué et les nombreux avantages des librairies de
    programmation distribués contrairement aux socket seuls qui sont bien plus
    complexes à utiliser. Il a aussi été très intéressant car il m'a permis de
    comprendre comment fontionne et une blockchain distribué et ca a été une
    formidable opportinuté d'en programmer une.
    
    

%\end{onehalfspace}

\end{document}
