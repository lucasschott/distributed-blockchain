\documentclass[a4paper,11pt,DIV=12]{scrreprt}
\usepackage[utf8]{inputenc}
\usepackage[T1]{fontenc}
\usepackage[francais]{babel}
\usepackage{graphicx}
\usepackage[export]{adjustbox}
\usepackage{setspace}
\usepackage{dirtytalk}
\usepackage{amsmath}
\usepackage[lined,boxed,commentsnumbered, ruled,vlined,linesnumbered, french]{algorithm2e}
\usepackage{algorithmic}
\usepackage{bytefield}
\usepackage{xcolor}
\usepackage{listings}
\definecolor{codegray}{rgb}{0.92,0.92,0.92}

\lstset{basicstyle=\ttfamily\small,
backgroundcolor=\color{codegray},   
showstringspaces=false,
commentstyle=\color{red},
keywordstyle=\color{blue}
}

\renewcommand{\algorithmicrequire}{\textbf{Require:}}
\renewcommand{\algorithmicensure}{\textbf{Ensure:}}
\renewcommand{\algorithmiccomment}[1]{\{#1\}}
\renewcommand{\algorithmicend}{\textbf{fin}}
\renewcommand{\algorithmicif}{\textbf{si}}
\renewcommand{\algorithmicthen}{\textbf{alors}}
\renewcommand{\algorithmicelse}{\textbf{sinon}}
\renewcommand{\algorithmicelsif}{\algorithmicelse\ \algorithmicif}
\renewcommand{\algorithmicendif}{\algorithmicend\ \algorithmicif}
\renewcommand{\algorithmicfor}{\textbf{pour}}
\renewcommand{\algorithmicforall}{\textbf{pour tout}}
\renewcommand{\algorithmicdo}{\textbf{faire}}
\renewcommand{\algorithmicendfor}{\algorithmicend\ \algorithmicfor}
\renewcommand{\algorithmicwhile}{\textbf{tant que}}

\title{Rapport de projet: Blockchain distribuée}
\author{Lucas SCHOTT}


\begin{document}

%\begin{onehalfspace}

    \maketitle

    \chapter{Introduction}

    \section{Présentation}

    Le but de ce projet était de réaliser une blockchain distibuée et de
    observer son évolution en simulant un cas d'utilisation de cette
    blockchain. Malheurseusement ce projet n'a pas abouti par manque de temps
    (du aux autre projets et examens que j'ai priorisés), et par un mauvais
    choix de technologie.

    Dans cette blockchain distribuée, les noeuds bloc stockent la blockchain,
    receptionnent des demandes de transactions et mine des block pour valider
    ces transactions.
    Les noeuds participants font des demandes de transactions, et peuvent
    demander aux noeuds bloc la quantité de points blockchain dont ils disposent.

    Ce projet à été réalisé en C++ en utilisant les sockets. Il est uniquement
    utilisable avec des adresse IPv6.
    \\

    \section{fonctionnement}

    Un participant peut contacter un noeud bloc et lui envoyer une demande de
    transacation d'un compte a un autre, ce noeud bloc stock met la transaction
    en attente. Le participant peut demander a un noeud bloc quelle est la
    quantité de points blockchain dont dispose un compte, c'est ce qu'on
    appelle la balance elle peut etre négative si un participant effectue une
    transaction supérieur à ce qu'il dispose.

    Les noeud blocs attendent d'avoir suffisement de transactions en attente
    pour ensuite miner un block et valider les transactions. Il lui faut 10
    transactions en attente pour commencer le minage. Une fois le block
    miné, le noeud bloc va l'envoyer à tous les autres noeuds bloc auquel il
    est connecté.
    \\

    Plusieurs noeuds bloc peuvent se connecter entre eux, au moment de la
    connection de deux blocs ceux-ci s'échangent les blockchain qu'ils
    possèdent. Le noeud bloc garde la blockchain valide la plus longue.
    Ensuite les noeuds bloc s'envoient mutuellement les listes des
    autres noeuds bloc qu'il connaissent, et les noeuds blocs vont ainsi démarrer de nouvelles
    procédure de connexion avec les autres noeuds bloc. Tous les noeaud blocs sont
    connectés entre eux avec une topologie full-mesh. Ainsi quand un noeud bloc
    mine un nouveau block, il va le transmettre à tous les autres noeuds bloc. 

    Les noeuds bloc connectés s'envoient des messages de manière périodique pour
    maintenir leur connection, si un noeud bloc ne reçoit plus de messages d'un
    autre noeud bloc il va le conséderer comme mort et les deux noeuds bloc se
    déconnecteront. Un noeud bloc est considéré comme mort au bout de 30 secondes
    sans réponse.
    \\

    Dans la pratique je n'ai pas réussi transférer une blockchain entière d'un
    noeud bloc à un autre, ainsi quand un nouveau noeud bloc, s'inscrit auprès
    des autres noeuds bloc, il n'arrive jamais à obtenir la blockchain. Et les
    échanges de block de fonctionneront pas.

    Donc dans l'état du projet auquel je suis il n'y a pas de distribution de
    la blockchain. On peut créer un noeud bloc, des noeuds participants peuvent
    lui envoyer des transactions, et obtenir leur balance. Mais malheureusement
    ce noeud bloc fait office de serveur central. On arrive néanmoins a
    connecter les noeuds blocs entre eux. Ils vont s'envoyer des message de
    maintient de connexion s'échanger les adresses des autres noeuds bloc, mais
    n'arrivent pas à s'échanger des données relatives à la blockchain.

    \chapter{Utilisation}

    \ \\ \\

    \begin{lstlisting}
    $ ./bloc <compte> <adresse_IPv6|nom_de_domaine> <numero_de_port> [-v]
    \end{lstlisting}

    Sert à creer un noeud bloc unique, avec un compte pour récupere les
    récompenses du à son travail de minage, en lui donnant aussi une adresse IPv6 à utiliser
    disponible sur la machine courante et un numéro de port à utiliser, 
    ce noeud bloc peut recevoir des demandes de transactions 
    \\ \\

    \begin{lstlisting}
    $ ./bloc <compte>  <adresse|nom_de_domaine> <numero_de_port>
    <adresse|nom_de_domaine> <numero_de_port> [-v]
    \end{lstlisting}

    Permet de creer un noeud bloc comme précédement et de le connecter à un autre
    noeud bloc déjà existant, en lui donnant en plus de sa propre adresse et
    numéro de son port, l'adresse (ou le nom de domaine) et
    le numéro de port du server auquel se connecter. Le nouveau noeud bloc
    communiquera avec celui déjà en place et intègrera le groupe de noeuds bloc
    dans lequel le noeud bloc contacté est déjà intégré.
    \\ \\

    \begin{lstlisting}
    $ ./participant <adresse_IPv6|nom_de_domaine> <port>
    <comtpe_source> <compte_destination> <montant>[-v]
    \end{lstlisting}

    Permet a un participant de contacter un noeud bloc en lui faisant une demande
    de transaction du compte source au compte destination, du montant donné.
    \\ \\

    \begin{lstlisting}
    $ ./participant  <adresse_IPv6|nom_de_domaine> <port>
    balance <compte> [-v]
    \end{lstlisting}

    Permet au noeud participant de demander au noeud bloc la balance du compte spécifié(nombre de points blockchain).
    \\ \\

    Les deux programmes, \emph{participant} comme \emph{bloc} peuvent prendre
    l'option \emph{-v} (verbose) en argument. Avec cette option activée le
    programme affiche explicitement ses différentes actions, et plus
    particulièrement les détails de synchonisations entre server, et les
    émissions et réceptions de messages.

    \chapter{Format de messages}


    Il y a cinq formats de messages:
    \begin{itemize}
        \item Le format \emph{transaction\_t} est utilisé quand un participant ou un noeud bloc envoi un
            transaction à un autre noeud bloc.
            \\
            \\

            \begin{bytefield}{32}
                \bitheader[lsb=0]{0,7,8,15,16,23,24,31,32} \\
                \bitbox{16}{2 octets: en-tête} & \bitbox[lrt]{16}{} \\
                \wordbox[lr]{1}{50 octets: compte source} \\
                \skippedwords \\
                \wordbox[lrb]{1}{} \\
                \wordbox[lr]{1}{50 octets: compte destination} \\
                \skippedwords \\
                \wordbox[lrb]{1}{} \\
                \wordbox[lrb]{1}{4 octets: montant de la transaction} \\
            \end{bytefield}

        \item Le format \emph{ask\_balance\_t} est utilisé quand un participant
            demande sa balance à un noeud bloc.
            \\
            \\

            \begin{bytefield}{32}
                \bitheader[lsb=0]{0,7,8,15,16,23,24,31,32} \\
                \bitbox{16}{2 octets: en-tête} & \bitbox[lrt]{16}{} \\
                \wordbox[lr]{1}{50 octets: compte} \\
                \skippedwords \\
                \wordbox[lrb]{1}{} \\
            \end{bytefield}

        \item Le format \emph{send\_balance\_t} est utilisé quand un noeud
            bloc envoi le montant de la balance d'un participant à celui ci.
            \\
            \\

            \begin{bytefield}{32}
                \bitheader[lsb=0]{0,7,8,15,16,23,24,31,32} \\
                \bitbox{16}{2 octets: en-tête} & \bitbox[lrtb]{16}{} \\
                \bitbox[lrb]{16}{4 octets: montant } \\
            \end{bytefield}

        \item Le format \emph{syn\_t} est utilisé quand un noeud
            bloc envoi des messages de connexion ou de synchronisation avec un
            autre noeud bloc.
            \\
            \\

            \begin{bytefield}{32}
                \bitheader[lsb=0]{0,7,8,15,16,23,24,31,32} \\
                \bitbox{16}{2 octets: en-tête} & \bitbox[lrt]{16}{} \\
                \wordbox[lr]{1}{16 octets: adresse noeud bloc } \\
                \skippedwords \\
                \wordbox[lrb]{1}{} \\
                \bitbox[lrb]{32}{4 octets: port} \\
            \end{bytefield}

        \item Et le format \emph{block\_t} est utilisé quand un noeud
            bloc envoi un block a un autre noeud bloc.
            \\
            \\

            \begin{bytefield}{32}
                \bitheader[lsb=0]{0,7,8,15,16,23,24,31,32} \\
                \bitbox{16}{2 octets: en-tête} & \bitbox[lrt]{16}{} \\
                \wordbox[lr]{1}{65 octets: hash de block précédant } \\
                \skippedwords \\
                \wordbox[lrb]{1}{} \\
                \wordbox[lrt]{1}{8 octets: timstamp} \\
                \wordbox[lrb]{1}{} \\
                \wordbox[lrtb]{1}{4 octets: nonce} \\
            \end{bytefield}



    \end{itemize}


    \chapter{Implémentation}


    \section{Échanges client/server}


    \subsection{Participant}

    \begin{algorithm}
        \begin{algorithmic}[H]
            \STATE {ouverture de la socket vers le noaeud bloc}
            \IF{le message est une transactinon}
                \STATE {envoyer les données au noeud bloc}
            \ELSIF{le message est une demande de balance}
                \STATE {envoyer la requête au noeud block}
                \STATE {attendre la réponse du noeud bloc pendant une seconde}
                \STATE {afficher la réponse}
            \ENDIF
        \end{algorithmic}
        \caption{Client: algorithme principal}
        \label{alg:client} 
    \end{algorithm}


    \subsection{Noeud bloc}
    
    \begin{algorithm}
        \begin{algorithmic}[H]
            \WHILE{le server est ouvert}
                \STATE{reception de message}
                \IF{le message est une transaction}
                    \STATE {mettre la transaction en attente}
                    \IF{il y a suffisament de messages en attente}
                        \STATE {miner un block}
                        \STATE {ajouter la block a la blockchain}
                        \STATE {envoyer le block au autres noeud bloc}
                    \ENDIF
                \ELSIF{le message est balance}
                    \STATE {parcourir la blockchain}
                    \STATE {envoyer le montant au client}
                \ENDIF
            \ENDWHILE
        \end{algorithmic}
        \caption{Server: réponse au client}
        \label{alg:server} 
    \end{algorithm}

    \newpage

    \section{Connexion des servers}

    \begin{algorithm}
        \begin{algorithmic}[H]
            \WHILE{le server est ouvert}
                \STATE{reception de message}
                \IF{le message est server connect}
                    \STATE{lui envoyer tous les servers auquels on est déjà
                    connecté}
                    \STATE{l'enregistrer dans la table des servers connectés et
                    initialiser la date de dernier keep\_alive}
                    \STATE{lui envoyer un message connect}
                \ELSIF{le message est keep}
                    \STATE{mettre a jour la date de dernier keep\_alive de ce server}
                \ELSIF{le message est share server}
                    \STATE{ajouter le server dans la table des servers connus
                    et initialiser la date de dernier keep\_alive}
                    \STATE{lui envoyer un message connect}
                \ENDIF
            \ENDWHILE
        \end{algorithmic}
        \caption{Server thread1: connexion entre servers}
        \label{alg:server} 
    \end{algorithm}
    
    \begin{algorithm}
        \begin{algorithmic}[H]
            \WHILE{le server est ouvert}
                \STATE{envoyer un message keep\_alive à tous les servers connectés}
                \WHILE{parcours de chaque server de la table des servers
                connus}
                    \IF{la date de dernier keep\_alive est ancienne de plus de
                    30 secondes}
                        \STATE{supprimer le server de la table}
                    \ENDIF
                \ENDWHILE
                \STATE attendre 5 secondes
            \ENDWHILE
        \end{algorithmic}
        \caption{Server thread2: maintient des connections avec les servers}
        \label{alg:server} 
    \end{algorithm}

    \chapter{Problèmes rencontrés et Conclusion}


    Ce projet à été très intéressant à réaliser, je m'y suis pris très tard à
    cause des nombreux autres travaux à réalier pendant ce semestre.
    Au commencement du projet, j'ai tendé de l'implémenter un java en utilisant
    la bibliothèque RM, je me suis rapidment confronté à un prolème car je ne
    comprennais comment envoyer un block, ou encore moins la blockchain entière
    à un autre noeud. Alors après avoir été resté bloqué de nombreuses heures,
    je me suis dit que je vais le faire en C++, en utilisant les socket, en
    m'inspirant du projet que j'avais réaliser le semestre dernier en cours en
    réseaux. J'ai bien avancer pendant plusieurs jours jusqu'à être bloqué au
    moment d'envoyer une blockchain entière, où la succession des messages à
    envoyer, un block après l'autre suivit de toutes les transactions du block,
    m'ont totalement empêché d'avancer. Je suis suis alors dit que j'avais fait
    un mauvais choix en passant sur le C++ avec les socket. Je suis repassé sur
    le projet en java/RMI et c'est à ce moment la que j'ai enfin réaliser
    comment envoyer des objets simplement en les sérialisant. Mais à ce moment
    la il était trop tard. Et je n'ai pas eu le temps de finir.
    J'ai donc fait en quelque sorte deux projets en parallèle, un en C++ avec
    les socket et un en java/RMI. Le projet le plus aboutit des deux est le
    projet en C++, c'est celui que j'ai détaillé dans ce rapport.\\
    \\
    
    Le sujet du projet était très intéréssant, mais je n'ai malheureusement pas
    pu en explorer tous les recoins. Il m'a néanmoins appris énormément sur le
    programmation distribué et les nombreux avantages des librairies de
    programmation distribués contrairement aux socket qui sont bien plus
    complexes à utiliser. Si je devais le refaire maintenant je le ferait d'une
    toute autre manière. Même si les résultats ne sont pas la je ressort quand
    même grandi de ce projet.
    
    

%\end{onehalfspace}

\end{document}
